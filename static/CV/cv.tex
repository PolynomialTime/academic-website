%%%%%%%%%%%%%%%%%%%%%%%%%%%%%%%%%%%%%%%%%
% Medium Length Professional CV
% LaTeX Template
% Version 2.0 (8/5/13)
%
% This template has been downloaded from:
% http://www.LaTeXTemplates.com
%
% Original author:
% Trey Hunner (http://www.treyhunner.com/)
%
% Important note:
% This template requires the resume.cls file to be in the same directory as the
% .tex file. The resume.cls file provides the resume style used for structuring the
% document.
%
%%%%%%%%%%%%%%%%%%%%%%%%%%%%%%%%%%%%%%%%%

%----------------------------------------------------------------------------------------
%	PACKAGES AND OTHER DOCUMENT CONFIGURATIONS
%----------------------------------------------------------------------------------------

\documentclass{resume} % Use the custom resume.cls style

\usepackage[left=1in,top=1in,right=1in,bottom=1in]{geometry} % Document margins
\usepackage{color}
\usepackage{hyperref}
\usepackage{fontawesome}
\usepackage[misc]{ifsym}
\hypersetup{hidelinks} 

\newcommand{\tab}[1]{\hspace{.2667\textwidth}\rlap{#1}}
\newcommand{\itab}[1]{\hspace{0em}\rlap{#1}}
\name{Yang Chen} % Your name
%\address{Ph.D. in Computer Science -- The University of Auckland -- Auckland, New Zealand}
\address{  NZ: (+64) 022-5250016 / CN: (+86) 185-1182-2753} % Your address
\address{\Letter ~ \href{mailto:yang.chen@auckland.ac.nz}{\texttt{yang.chen@auckland.ac.nz}} ~~ $\bullet$ ~~ \faGlobe ~ \url{www.yangchen.info} %~~ $\bullet$ ~~ \faGithub ~  \href{https://github.com/PolynomialTime}{\texttt{PolynomialTime}}
}% Your phone number and email
%\address{123 Pleasant Lane \\ City, State 12345} % Your secondary addess (optional)
\begin{document}

%----------------------------------------------------------------------------------------
%	EDUCATION SECTION
%----------------------------------------------------------------------------------------

\begin{rSection}{EDUCATION}

\begin{rSubsection}{The University of Auckland}{Auckland, New Zealand}{Ph.D. Candidate, School of Computer Science}{November 2018 -- April 2022}%October 2021 (expected)}
\item Supervisors: Jiamou Liu and Bakh Khoussainov
\item Thesis: {\em From One to Infinity: New Algorithms for Reinforcement Learning and Inverse Reinforcement Learning}
\end{rSubsection}

\begin{rSubsection}{The University of Auckland}{Auckland, New Zealand}{First Class Honours in Computer Science}{July 2017 -- July 2018}
%\item Cumulative GPA: A (A+)
\item Dissertation: {\em Network Building: Methodological Foundations and Algorithmic Analysis}
\end{rSubsection}

\begin{rSubsection}{Beijing Institute of Technology}{Beijing, China}{Bsc in Computer Science \& Technology}{August 2013 -- June 2017}
\end{rSubsection}
%Minor in Linguistics \smallskip \\
%Member of Eta Kappa Nu \\
%Member of Upsilon Pi Epsilon \\
\end{rSection}


%----------------------------------------------------------------------------------------
%	INTERNSHIP SECTION
%----------------------------------------------------------------------------------------
\begin{rSection}{Work Experience}
\begin{rSubsection}{The University of Auckland}{Auckland, New Zealand}{Research Fellow}{June 2021 -- Present}
\item I am currently affiliated with the University of Auckland as a research fellow. %The paper is going to be published and a prototype is ready to be deployed to support online business. 
\end{rSubsection}
	\begin{rSubsection}{Alibaba DAMO Academy}{Beijing, China}{Research Intern}{September 2020 -- January 2021}
\item I served as a principal contributor and programmer of a research project, where I proposed a novel framework that combines deep learning and bandit techniques to enhance the efficiency and accuracy of the recommender system. %The paper is going to be published and a prototype is ready to be deployed to support online business. 
\end{rSubsection}
\end{rSection}



%----------------------------------------------------------------------------------------
%	RESEARCH SECTION
%----------------------------------------------------------------------------------------
\begin{rSection}{Research Interests}
My main research interest lies in (deep) reinforcement learning in multi-agent systems. I aim to solve issues modelled as multi-agent systems from the learning and game-theoretical perspective. Along this line, my ultimate goal is to reinforce the insights into rules of applying reinforcement learning in conjunction with game theory. Very recently, my focus moves to reinforcement learning in games with tons of agents. I attempt to explore exciting results in such scenarios through combining reinforcement learning and the mean field theory.
%\begin{tabular}{ @{} >{\bfseries}l @{\hspace{6ex}} l }
%Computer Languages &  C/C++, MATLAB \\
%Software \& Tools & HTML, LaTeX, Excel, Gerris, Mathematica, ASPEN Plus, Tecplot \\
%\end{tabular}
\end{rSection}


%----------------------------------------------------------------------------------------
%	TEACHING, SUPERVISION & SERVICE SECTION
%----------------------------------------------------------------------------------------

\begin{rSection}{Supervision}
	\begin{rSubsection}{Honours}{}{}{}
		\item {\bf Yiwei Qi.}\\ The University of Auckland. \hfill {\em February 2022 -- November 2022} \\Topic: {\em Building A Game-Playing Agent Using Decision Theory}
	\end{rSubsection}
\end{rSection}

\begin{rSection}{Teaching}
	\begin{rSubsection}{}{}{}{}
		\item {\bf COMPSCI 220: Algorithms and Data Structures}\\ Guest Lecturer, The University of Auckland. \hfill {\em Semester 1, 2022} 
		\item {\bf COMPSCI 399 Capstone: Computer Science}\\ Project Supervisor, The University of Auckland. \hfill {\em Semester 2, 2021}
	\end{rSubsection}
\end{rSection}

\begin{rSection}{Academic Services}
	\begin{rSubsection}{}{}{}{}
		\item {\bf Conference Reviewer:} ICNLP 2022, BSCI 2022. 		
		\item {\bf Journal Reviewer:} Social Network Analysis and Mining (SNAM).

	\end{rSubsection}
\end{rSection}


\begin{rSection}{Awards}
\begin{rSubsection}{}{}{}{}
\item Google Global PhD Fellowship Nomination (Austrilia \& New Zealand) \hfill {\em August 2020}
\item Best Paper Award, BSCI 2019. \hfill {\em July 2019}
\item Summer Scholarship Funding from PDH Research Partnership. \hfill {\em November 2018}
\item University of Auckland Doctoral Scholarship (approx. NZD \$85000). \hfill {\em October 2018}
\end{rSubsection}
\end{rSection}

%----------------------------------------------------------------------------------------
%	PUBLICATIONS
%----------------------------------------------------------------------------------------
\begin{rSection}{SELECTED PUBLICATIONS}
\begin{rSubsection}{PREPRINTS}{}{}{}
	\item {\bf Adversarial Inverse Reinforcement Learning for Mean Field Games}\\
		\textbf{Yang Chen}, Jiamou Liu, Bakhadyr Khoussainov. \textbf{arXiv 2021.}
\end{rSubsection}
\begin{rSubsection}{CONFERENCES}{}{}{}
	\item {\bf  Interconnected Neural Linear Contextual Bandits with UCB Exploration}\\
		\textbf{Yang Chen}, Miao Xie, Jiamou Liu, Kaiqi Zhao. {\em 26th Pacific-Asia Conference on Knowledge Discovery and Data Mining.} \textbf{PAKDD 2022.} (Core A, CCF C)
	\item {\bf Individual-Level Inverse Reinforcement Learning for Mean Field Games}\\
		\textbf{Yang Chen}, Libo Zhang, Jiamou Liu, Shuyue Hu. {\em The 21st International Conference on Autonomous Agents and Multi-agent Systems.} \textbf{AAMAS 2022.} (Core A*, CCF B)
	\item {\href{http://www.ifaamas.org/Proceedings/aamas2020/pdfs/p1807.pdf}{\bf Social Capital Games as A Framework for Social Structural Pattern Emergence}}\\ 
		\textbf{Yang Chen}, Jiamou Liu. {\em IEEE/ACM International Conference on Advances in Social Networks Analysis and Mining.} \textbf{ASONAM 2020.}
	\item {\href{http://www.ifaamas.org/Proceedings/aamas2020/pdfs/p1807.pdf}{\bf Social Structure Emergence: A Multi-agent Reinforcement Learning Framework for Relationship Building}}\\ 
		\textbf{Yang Chen}, Jiamou Liu, He Zhao, Hongyi Su. {\em The 19th International Conference on Autonomous Agents and Multi-agent Systems.} \textbf{AAMAS 2020.} (Core A*, CCF B)
	\item {\href{https://github.com/PolynomialTime/ASONAM2019/blob/master/asonam2019.pdf}{\bf Becoming Gatekeepers Together with Allies: Collaborative Brokerage over Social Networks}}\\
		\textbf{Yang Chen}, Jiamou Liu. {\em The 2019 IEEE/ACM International Conference on Advances in Social Networks Analysis and Mining.} \textbf{ASONAM 2019.}
	\item {\href{https://github.com/PolynomialTime/WISE2017/blob/master/Can_Reinforcement_Learning_Enhance_Social_Capital_.pdf}{\bf A Reinforcement Learning Approach to Gaining Social Capital with Partial Observation}}\\
		He Zhao, Hongyi Su, \textbf{Yang Chen (contact)}, Jiamou Liu, Hong Zheng, Bo Yan. {\em The 16th Pacific Rim International Conference on Artificial Intelligence.} \textbf{PRICAI 2019.} (Core A, CCF C)
	\item {\href{https://github.com/PolynomialTime/WISE2017/blob/master/WISE2017.pdf}{\bf Dynamic Relationship Building: Exploitation Versus Exploration on a Social Network}}\\
		Bo Yan, \textbf{Yang Chen}, Jiamou Liu. {\em The 18th International Conference on Web Information Systems Engineering.} \textbf{WISE 2017.} (Core A, CCF C)
\end{rSubsection}
\begin{rSubsection}{WORKSHOPS}{}{}{}
\item {\href{https://link.springer.com/chapter/10.1007/978-981-15-3281-8_14}{\bf Can Reinforcement Learning Enhance Social Capital?}}\\
		He Zhao, Hongyi Su, \textbf{Yang Chen}, Jiamou Liu, Bo Yan, Hong Zheng. {\em The International Workshop on Web Information Systems in the Era of AI.} 
\item {\href{https://www.researchgate.net/profile/Yang-Chen-67/publication/334358789_Distributed_Community_Detection_over_Blockchain_Networks_Based_on_Structural_Entropy/links/5d281b20299bf1547cadb905/Distributed-Community-Detection-over-Blockchain-Networks-Based-on-Structural-Entropy.pdf}{\bf Distributed Community Detection over Blockchain Networks Based on Structural Entropy}}\\
		\textbf{Yang Chen}, Jiamou Liu. {\em The 2019 ACM International Symposium on Blockchain and Secure Critical Infrastructure.} \textbf{BSCI 2019. \textcolor{red}{(Best Paper Award)}} 
\end{rSubsection}

\end{document}
