%%%%%%%%%%%%%%%%%%%%%%%%%%%%%%%%%%%%%%%%%
% Medium Length Professional CV
% LaTeX Template
% Version 2.0 (8/5/13)
%
% This template has been downloaded from:
% http://www.LaTeXTemplates.com
%
% Original author:
% Trey Hunner (http://www.treyhunner.com/)
%
% Important note:
% This template requires the resume.cls file to be in the same directory as the
% .tex file. The resume.cls file provides the resume style used for structuring the
% document.
%
%%%%%%%%%%%%%%%%%%%%%%%%%%%%%%%%%%%%%%%%%

%----------------------------------------------------------------------------------------
%	PACKAGES AND OTHER DOCUMENT CONFIGURATIONS
%----------------------------------------------------------------------------------------

\documentclass{resume} % Use the custom resume.cls style

\usepackage[left=1in,top=1in,right=1in,bottom=1in]{geometry} % Document margins
\usepackage{color}
\usepackage{hyperref}
\usepackage{fontawesome}
\usepackage[misc]{ifsym}
\hypersetup{hidelinks} 
\usepackage{CJK}

\begin{CJK}{UTF8}{gbsn}
\newcommand{\tab}[1]{\hspace{.2667\textwidth}\rlap{#1}}
\newcommand{\itab}[1]{\hspace{0em}\rlap{#1}}
\name{陈阳} % Your name
%\address{Ph.D. in Computer Science -- The University of Auckland -- Auckland, New Zealand}
\address{  NZ: (+64) 022-5250016 / CN: (+86) 185-1182-2753} % Your address
\address{\Letter ~ \href{mailto:yang.chen@auckland.ac.nz}{\texttt{yang.chen@auckland.ac.nz}} ~~ $\bullet$ ~~ \faGlobe ~ \href{https://www.yangchen.info}{\texttt{www.yangchen.info}} %~~ $\bullet$ ~~ \faGithub ~  \href{https://github.com/PolynomialTime}{\texttt{PolynomialTime}}
}% Your phone number and email
%\address{123 Pleasant Lane \\ City, State 12345} % Your secondary addess (optional)
\begin{document}

%----------------------------------------------------------------------------------------
%	EDUCATION SECTION
%----------------------------------------------------------------------------------------

\begin{rSection}{教育背景}

\begin{rSubsection}{University of Auckland}{奥克兰, 新西兰}{博士,计算机科学}{2018.11 -- 2022.09}%October 2021 (expected)}
\item 导师: Jiamou Liu 和 Bakhadyr Khoussainov
\item 论文: {\em From One to Infinity: New Algorithms for Reinforcement Learning and Inverse Reinforcement Learning}
\end{rSubsection}

\begin{rSubsection}{University of Auckland}{奥克兰, 新西兰}{一等荣誉学士,计算机科学}{2017.07 -- 2018.07}
%\item Cumulative GPA: A (A+)
\item 论文: {\em Network Building: Methodological Foundations and Algorithmic Analysis}
\end{rSubsection}

\begin{rSubsection}{北京理工大学}{北京, 中国}{学士,计算机科学与技术}{2013.08 -- 2017.06}
\end{rSubsection}
%Minor in Linguistics \smallskip \\
%Member of Eta Kappa Nu \\
%Member of Upsilon Pi Epsilon \\
\end{rSection}


%----------------------------------------------------------------------------------------
%	INTERNSHIP SECTION
%----------------------------------------------------------------------------------------
\begin{rSection}{工作经历}
\begin{rSubsection}{University of Auckland}{奥克兰, 新西兰}{博后研究员}{2022.09 -- 至今}
%\item I am currently a research fellow affiliated with the School of Computer Science at The University of Auckland. 
\end{rSubsection}
	\begin{rSubsection}{阿里巴巴达摩院}{北京, 中国}{研究实习生}{2020.09 -- 2021.07}
%\item I served as a principal contributor and programmer of a research project. I proposed a novel framework that combines deep learning and bandit techniques to enhance the efficiency and accuracy of the recommender system. The performance bound was derived using statistical learning tools. The paper has been published, and a prototype is ready to be deployed to support the online business. 
\end{rSubsection}
\end{rSection}



%----------------------------------------------------------------------------------------
%	RESEARCH SECTION
%----------------------------------------------------------------------------------------
\begin{rSection}{研究兴趣}

我的研究兴趣包括强化学习、多智能体系统和博弈论。近期,我的目标是从强化学习和博弈论的角度解决建模为多智能体系统的问题,我的最终目标是加强对将强化学习与博弈论相结合的模型的认识与应用。​最近,我的重点转移到应用强化学习和大型语言模型在推理,行为建模,以及道德人工智能和负责任的人工智能中的应用。

%My main research interest lies in the algorithmic game theory, where I aim to design game solvers using probabilistic computation, statistics and information theory.
%Recently, I have aimed to solve issues modelled as multi-agent systems from the learning-theoretic and game-theoretic perspectives. Along this line, my ultimate goal is to reinforce the insights into theories of applying game theory in conjunction with reinforcement learning. Very lately, my focus has moved to the theory of reinforcement learning in games with massive agents. I attempt to explore exciting results in such scenarios by combining reinforcement learning and the mean-field theory. In addition to the research interest, in theory, I am also working on applying reinforcement learning in natural language processing and automatic reasoning. 
%\begin{tabular}{ @{} >{\bfseries}l @{\hspace{6ex}} l }
%Computer Languages &  C/C++, MATLAB \\
%Software \& Tools & HTML, LaTeX, Excel, Gerris, Mathematica, ASPEN Plus, Tecplot \\
%\end{tabular}
\end{rSection}


%----------------------------------------------------------------------------------------
%	TEACHING, SUPERVISION & SERVICE SECTION
%----------------------------------------------------------------------------------------
\begin{rSection}{教学}
	\begin{rSubsection}{Postgraduate Courses}{}{}{}
	\item {\bf COMPSCI 713: Artificial Intelligence Foundations}\\ Lecturer, University of Auckland. \hfill {\em Semester 1, 2024}
	\item {\bf COMPSCI 761: Advanced Topics in Artificial Intelligence}\\ Lecturer, University of Auckland. \hfill {\em Semester 2, 2022 -- 2023}
	\item {\bf COMPSCI 769: Natural Language Processing}\\ Guest Lecturer, University of Auckland. \hfill {\em Semester 2, 2024} 
		%\item {\bf COMPSCI 399 Capstone: Computer Science}\\ Project Supervisor, University of Auckland. \hfill {\em Semester 2, 2021}
	\end{rSubsection}
	\begin{rSubsection}{Undergraduate Courses}{}{}{}
		\item {\bf COMPSCI 367: Artificial Intelligence}\\ Guest Lecturer, University of Auckland. \hfill {\em Semester 2, 2024}
		\item {\bf COMPSCI 220: Algorithms and Data Structures}\\ Guest Lecturer, University of Auckland. \hfill {\em Semester 1, 2022}
	\end{rSubsection}
\end{rSection}

%\begin{rSection}{学生指导}
%One BSc (Honours) Student (graduated); four master students; one PhD Student (mentoring).
	%\begin{rSubsection}{BSc (Honours) Students}{}{}{}
		%\item {\bf Yiwei Qi}\\ University of Auckland. \hfill {\em February 2022 -- November 2022} \\Topic: {\em Building A Game-Playing Agent Using Decision Theory}
	%\end{rSubsection}
	%\begin{rSubsection}{Master Students}{}{}{}
		%\item {\bf Yitan Zhang, Hao Zhong, Junyi Yang, Shaiyu Chen}\\ University of Auckland. \hfill {\em September 2023 -- Present} \\Topic: {\em Reinforcement Learning in Large-scale Multi-agent Systems}
	%\end{rSubsection}
	%\begin{rSubsection}{Ph.D. Students}{}{}{}
		%\item {\bf Libo Zhang (mentoring)}\\ University of Auckland \hfill {\em November 2021 -- Present} \\Topic: {\em Learning Correlated Equilibria in Multi-player Games}
	%\end{rSubsection}
%\end{rSection}

%----------------------------------------------------------------------------------------
%	GRANTS SECTION
%----------------------------------------------------------------------------------------
\begin{rSection}{项目基金}
	\begin{rSubsection}{当前项目}{}{}{}
		\item {\bf AI-based behavioural analytics for live sports broadcast} \hfill {\em 2024 -- 2026}\\
			{\em 副研究员}\\
			主研究员: Patrice Delmas\\
			资助机构: 新西兰商业、创新和就业部(MBIE)探索基金\\
			资助额度: 一百万新西兰元
	\end{rSubsection}
\end{rSection}


%----------------------------------------------------------------------------------------
%	SERVICE & AWARDS SECTION
%----------------------------------------------------------------------------------------

\begin{rSection}{学术活动组织}
	\begin{rSubsection}{}{}{}{}
		\item \href{https://www.aamas2024-conference.auckland.ac.nz/organization/organizing-committee/}{AAMAS 2024 local co-chair.}
		\item \href{https://www.aamas2024-conference.auckland.ac.nz/organization/organizing-committee/}{AAMAS 2024 AAAI track co-chair.}
	\end{rSubsection}
\end{rSection}

\begin{rSection}{学术服务}
	\begin{rSubsection}{}{}{}{}
		\item {\bf 会议审稿:} ACL 2024, EACI 2024, AAMAS 2023-2024, ICNLP 2022, BSCI 2022.
		\item {\bf 期刊审稿:} JMLR, SNAM.
	\end{rSubsection}
\end{rSection}


\begin{rSection}{获奖}
\begin{rSubsection}{}{}{}{}
\item AAMAS 2022 会议奖学金 \hfill {\em 2022.04}
\item 谷歌全球博士奖研金提名 (澳大利亚和新西兰区) \hfill {\em 2020.08}
\item 最佳论文奖, BSCI 2019. \hfill {\em 2019.07}
\item PDH Research Partnership暑期研究奖学金. \hfill {\em 2018.11}
\item 奥克兰大学博士奖学金. \hfill {\em 2018.10}
\end{rSubsection}
\end{rSection}

%----------------------------------------------------------------------------------------
%	PUBLICATIONS
%----------------------------------------------------------------------------------------
\begin{rSection}{发表著作}
\begin{rSubsection}{\large\em $\bullet$ (Multi-agent) Reinforcement Learning \& Imitation Learning}{}{}{}
	\item {\bf Meta-Inverse Reinforcement Learning for Mean Field Games with Probabilistic Context Variables}\\
		\textbf{\bf Yang Chen}, Xiao Lin, Bo Yan, Libo Zhang, Jiamou Liu, Neset \"{O}zkan Tan, Michael Witbrock. {\em Thirty-Eighth AAAI Conference on Artificial Intelligence.} \textbf{AAAI 2024.} (Core A*)\\
	\item {\bf Multi-Agent, Human-Agent and Beyond: A Survey on Cooperation in Social Dilemmas.} Hao Guo, Chunjiang Mu, \textbf{Yang Chen}, Chen Shen, Shuyue Hu, Zhen Wang. \textbf{Neurocomputing. 2024.}
	
	\item {\bf Adversarial Inverse Reinforcement Learning for Mean Field Games}\\
		\textbf{Yang Chen}, Libo Zhang, Zhenyun Deng, Neset \"{O}zkan Tan, Jiamou Liu, Michael Witbrock. {\em The 22nd International Conference on Autonomous Agents and Multi-agent Systems.} \textbf{AAMAS 2023.} (Core A*)\\
	\item {\bf Density-based Correlated Equilibrium for Markov Games.}\\
		Libo Zhang, \textbf{Yang Chen (co-first \& contact)}, Toru Takisaka, Bakh Khoussainov, Michael Witbrock, Jiamou Liu. {\em The 22nd International Conference on Autonomous Agents and Multi-agent Systems.} \textbf{AAMAS 2023.} (Core A*)\\
	\item {\href{https://ifaamas.org/Proceedings/aamas2022/pdfs/p253.pdf}{\bf Individual-Level Inverse Reinforcement Learning for Mean Field Games}}\\
		\textbf{Yang Chen}, Libo Zhang, Jiamou Liu, Shuyue Hu. {\em The 21st International Conference on Autonomous Agents and Multi-agent Systems.} \textbf{AAMAS 2022.} (Core A*)\\
	\item {\href{}{\bf  Interconnected Neural Linear Contextual Bandits with Upper Confidence Bound Exploration}}\\
		\textbf{Yang Chen}, Miao Xie, Jiamou Liu, Kaiqi Zhao. {\em 26th Pacific-Asia Conference on Knowledge Discovery and Data Mining.} \textbf{PAKDD 2022.} \\
	\item {\href{http://www.ifaamas.org/Proceedings/aamas2020/pdfs/p1807.pdf}{\bf Social Structure Emergence: A Multi-agent Reinforcement Learning Framework for Relationship Building}}\\ 
		\textbf{Yang Chen}, Jiamou Liu, He Zhao, Hongyi Su. {\em The 19th International Conference on Autonomous Agents and Multi-agent Systems.} \textbf{AAMAS 2020.} (Core A*)\\
	\item {\href{}{\bf Social Capital Games as A Framework for Social Structural Pattern Emergence}}\\ 
		\textbf{Yang Chen}, Jiamou Liu. {\em IEEE/ACM International Conference on Advances in Social Networks Analysis and Mining.} \textbf{ASONAM 2020.}\\
\end{rSubsection}

\begin{rSubsection}{\large\em $\bullet$ Multi-agent Behaviour Modelling \& Simulation}{}{}{}
	\item {\bf Behaviour Modelling of Social Animals via Causal Structure Discovery and Graph Neural Networks}\\
		Ga\"el Gendron (co-first), \textbf{Yang Chen (co-first)}, Mitchell Rogers, Yiping Liu, Mihailo Azhar, Shahrokh Heidari, David Arturo Soriano Valdez, Kobe Knowles, Padriac O'Leary, Simon Eyre, Michael Witbrock, Gillian Dobbie, Jiamou Liu and Patrice Delmas. {\em  The 23rd International Conference on Autonomous Agents and Multi-agent Systems.} \textbf{AAMAS 2024.} (Core A*)\\
	\item {\bf Meerkat Behaviour Recognition Dataset}\\Mitchell Rogers, Gaël Gendron, David Soriano Valdez, Mihailo Azhar, \textbf{Yang Chen}, Shahrokh Heidari, Caleb Perelini, Padriac O'leary, Kobe Knowles, Izak Tait, Simon Eyre, Michael Witbrock, Patrice Delmas. {\em 3rd Workshop on {\bf CV4Animals}: Computer Vision for Animal Behavior Tracking and Modeling (in conjunction with {\bf CVPR 2023})}.\\
	\item {
	\href{}{\bf MSDC: Non-intrusive Load Monitoring with a Dual-CNN Model}}\\
		Jialing He, Jiamou Liu, Zijian Zhang, \textbf{Yang Chen}, Yiwei Liu, Bakh Khoussainov, Liehuang Zhu. {\em Thirty-Seventh AAAI Conference on Artificial Intelligence.} \textbf{AAAI 2023.} (Core A*)\\
\end{rSubsection}

\begin{rSubsection}{\large\em $\bullet$ Natural Language Processing}{}{}{}
	\item {\bf Neuromodulation Gated Transformer.}\\
Kobe Knowles, Joshua Bensemann, Diana Benavides Prado, Vithya Yogarajan, Michael Witbrock, Gillian Dobbie, \textbf{Yang Chen}. {\em The Eleventh International Conference on Learning Representations.} {\bf ICLR 2023} Tiny Papers.\\
	\item {\bf Multi2Claim: Generating Scientific Claims from Multi-Choice Questions for Scientific Fact-Checking.}\\
Neset Tan, Trung Nguyen, Josh Bensemann, Alex Peng, Qiming Bao, \textbf{Yang Chen}, Mark Gahegan, Michael Witbrock. {\em The 17th Conference of the European Chapter of the Association for Computational Linguistics.} {\bf EACL 2023.}\\
	\item {\bf Contrastive Learning with Logic-driven Data Augmentation for Logical Reasoning over Text.}\\
Qiming Bao, Alex Yuxuan Peng, Zhenyun Deng, Wanjun Zhong, Neset Tan, Nathan Young, \textbf{Yang Chen}, Yonghua Zhu, Michael Witbrock, Jiamou Liu. {\em Symposium on Large Language Models \@IJCAI'23.} {\bf LLM@IJCAI'23.}\\
	\item {\bf Interpretable AMR-Based Question Decomposition for Multi-hop Question Answering.}\\
		Zhenyun Deng, Yonghua Zhu, \textbf{Yang Chen}, Michael Witbrock, Patricia Riddle. {\em The 31st International Joint Conference on Artificial Intelligence.} \textbf{IJCAI 2022.} (Core A*)\\
	\item {\bf Prompt-based Conservation  Learning for Multi-hop Question Answering.}\\
Zhenyun Deng, Yonghua Zhu, \textbf{Yang Chen}, Qianqian Qi, Michael Witbrock, Patricia Riddle. {\em The 29th International Conference on Computational Linguistics.} \textbf{COLING 2022.} (Core A)\\
\item {\href{https://advancesincognitivesystems.github.io/acs2021/data/ACS-21_paper_22.pdf}{\bf An explainability analysis of a sentiment prediction task using a transformer-based attention filter}}\\
		Neset \"{O}zkan Tan, Joshua Bensemann, Diana Benavides-Prado, \textbf{Yang Chen}, Mark Gahegan, Lia Lee, Alex Yuxuan Peng, Patricia Riddle, Michael Witbrock. {\em The Ninth Annual Conference on Advances in Cognitive Systems.} \textbf{ACS 2021.}\\
\end{rSubsection}


\begin{rSubsection}{\large\em $\bullet$ Graph Theory and Graph Neural Networks}{}{}{}
	\item {\bf Graph Transformer against Graph Perturbation by Flexible-pass Filter}\\
		Yonghua Zhu, Jincheng Huang, \textbf{Yang Chen}, Robert Amor, Michael Witbrock. {\em Journal of Information Fusion.} {\bf 2024.}\\
	\item {\bf Robust Node Classification on Graph Data with Graph and Label Noise}\\
		Yonghua Zhu, Lei Feng, Zhenyun Deng, \textbf{Yang Chen}, Robert Amor, Michael Witbrock. {\em Thirty-Eighth AAAI Conference on Artificial Intelligence.} \textbf{AAAI 2024.} (Core A*)\\
	\item {\href{}{\bf Efficient Size-Prescribed $k$-Core Search}}\\
		Yiping Liu, Bo Yan, Bo Zhao, Hongyi Su, \textbf{Yang Chen}, Michael Witbrock. {\em The 2023 IEEE/ACM International Conference on Advances in Social Networks Analysis and Mining.} \textbf{ASONAM 2023.}\\
	\item{
	\href{}{\bf Chain of Propagation Prompting for Node Classification}}\\
	Yonghua Zhu, Zhenyun Deng, \textbf{Yang Chen}, Robert Amor, Michael Witbrock. {\em ACM MultiMedia 2023.} \textbf{ACM MM 2023.} (Core A*)\\
	\item {\href{https://link.springer.com/chapter/10.1007/978-3-030-29908-8_9}{\bf A Reinforcement Learning Approach to Gaining Social Capital with Partial Observation}}\\
		He Zhao, Hongyi Su, \textbf{Yang Chen}, Jiamou Liu, Hong Zheng, Bo Yan. {\em The 16th Pacific Rim International Conference on Artificial Intelligence.} \textbf{PRICAI 2019.} \\
	\item {\href{https://github.com/PolynomialTime/WISE2017/blob/master/WISE2017.pdf}{\bf Dynamic Relationship Building: Exploitation Versus Exploration on a Social Network}}\\
		Bo Yan, \textbf{Yang Chen}, Jiamou Liu. {\em The 18th International Conference on Web Information Systems Engineering.} \textbf{WISE 2017.}\\
	\item {\href{https://link.springer.com/chapter/10.1007/978-981-15-3281-8_14}{\bf Can Reinforcement Learning Enhance Social Capital?}}\\
		He Zhao, Hongyi Su, \textbf{Yang Chen}, Jiamou Liu, Bo Yan, Hong Zheng. {\em The International Workshop on Web Information Systems in the Era of AI. {\bf 2019}.}\\
	\item {\href{https://www.researchgate.net/profile/Yang-Chen-67/publication/334358789_Distributed_Community_Detection_over_Blockchain_Networks_Based_on_Structural_Entropy/links/5d281b20299bf1547cadb905/Distributed-Community-Detection-over-Blockchain-Networks-Based-on-Structural-Entropy.pdf}{\bf Distributed Community Detection over Blockchain Networks Based on Structural Entropy}}\\
		\textbf{Yang Chen}, Jiamou Liu. {\em The 2019 ACM International Symposium on Blockchain and Secure Critical Infrastructure.} \textbf{BSCI 2019. \textcolor{red}{(Best Paper Award)}}\\
	\item {\href{https://github.com/PolynomialTime/ASONAM2019/blob/master/asonam2019.pdf}{\bf Becoming Gatekeepers Together with Allies: Collaborative Brokerage over Social Networks}}\\
		\textbf{Yang Chen}, Jiamou Liu. {\em The 2019 IEEE/ACM International Conference on Advances in Social Networks Analysis and Mining.} \textbf{ASONAM 2019.}\\
\end{rSubsection}
\end{CJK}


\end{document}
